% The document class supplies options to control rendering of some standard
% features in the result.  The goal is for uniform style, so some attention 
% to detail is *vital* with all fields.  Each field (i.e., text inside the
% curly braces below, so the MEng text inside {MEng} for instance) should 
% take into account the following:
%
% - author name       should be formatted as "FirstName LastName"
%   (not "Initial LastName" for example),
% - supervisor name   should be formatted as "Title FirstName LastName"
%   (where Title is "Dr." or "Prof." for example),
% - degree programme  should be "BSc", "MEng", "MSci", "MSc" or "PhD",
% - dissertation title should be correctly capitalised (plus you can have
%   an optional sub-title if appropriate, or leave this field blank),
% - thesis type should be formatted as one of the following:
%   * for the MEng degree programme either "enterprise" or "research" to
%     reflect the stream,
%   * for the MSc  degree programme "$X/Y/Z$" for a project deemed to be
%     X%, Y% and Z% of type I, II and III.
% - year              should be formatted as a 4-digit year of submission
%   (so 2014 rather than the accademic year, say 2013/14 say).

\documentclass[ % the name of the author
                    author={Alexandra Emily Moylett},
                % the name of the supervisor
                supervisor={Dr. Peter S. Turner and Prof. Noah Linden},
                % the degree programme
                    degree={PhD},
                % the thesis    title (which cannot be blank)
                     title={The Quantum-Classical Race},
                % the thesis subtitle (which can    be blank)
                  subtitle={Near-term quantum and classical algorithms for combinatorially hard problems},
                % the year of submission
                      year={\year} ]{templates/thesis}

\begin{document}

% =============================================================================

% After the front matter comes a number of chapters; under each chapter,
% sections, subsections and even subsubsections are permissible.  The
% pages in this part are numbered with Arabic numerals.  Note that:
%
% - A reference point can be marked using \label{XXX}, and then later
%   referred to via \ref{XXX}; for example Chapter\ref{chap:context}.
% - The chapters are presented here in one file; this can become hard
%   to manage.  An alternative is to save the content in seprate files
%   the use \input{XXX} to import it, which acts like the #include
%   directive in C.

\mainmatter

% -----------------------------------------------------------------------------

\chapter{Conclusion}
\label{chp:conclusion}

\section{Open Questions}

\subsection{Applications direction}

\subsection{Architecture direction}


% =============================================================================

% Finally, after the main matter, the back matter is specified.  This is
% typically populated with just the bibliography.  LaTeX deals with these
% in one of two ways, namely
%
% - inline, which roughly means the author specifies entries using the 
%   \bibitem macro and typesets them manually, or
% - using BiBTeX, which means entries are contained in a separate file
%   (which is essentially a databased) then inported; this is the 
%   approach used below, with the databased being thesis.bib.
%
% Either way, the each entry has a key (or identifier) which can be used
% in the main matter to cite it, e.g., \cite{X}, \cite[Chapter 2}{Y}.

\backmatter

\bibliography{introduction/introduction, preliminary_q_c/preliminary_q_c, preliminary_bs/preliminary_bs, tsp/tsp, noisy_circuit/noisy_circuit, classical_sim/classical_sim}

% -----------------------------------------------------------------------------

% The thesis concludes with a set of (optional) appendicies; these are 
% the same as chapters in a sense, but once signaled as being appendicies via
% the associated macro, LaTeX manages them appropriatly.

\appendix

% =============================================================================

\end{document}
