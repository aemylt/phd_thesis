% The document class supplies options to control rendering of some standard
% features in the result.  The goal is for uniform style, so some attention 
% to detail is *vital* with all fields.  Each field (i.e., text inside the
% curly braces below, so the MEng text inside {MEng} for instance) should 
% take into account the following:
%
% - author name       should be formatted as "FirstName LastName"
%   (not "Initial LastName" for example),
% - supervisor name   should be formatted as "Title FirstName LastName"
%   (where Title is "Dr." or "Prof." for example),
% - degree programme  should be "BSc", "MEng", "MSci", "MSc" or "PhD",
% - dissertation title should be correctly capitalised (plus you can have
%   an optional sub-title if appropriate, or leave this field blank),
% - dissertation type should be formatted as one of the following:
%   * for the MEng degree programme either "enterprise" or "research" to
%     reflect the stream,
%   * for the MSc  degree programme "$X/Y/Z$" for a project deemed to be
%     X%, Y% and Z% of type I, II and III.
% - year              should be formatted as a 4-digit year of submission
%   (so 2014 rather than the accademic year, say 2013/14 say).

\documentclass[ % the name of the author
                    author={Dominic Joseph Moylett},
                % the name of the supervisor
                supervisor={Dr. Peter S. Turner and Prof. Noah Linden},
                % the degree programme
                    degree={PhD},
                % the dissertation    title (which cannot be blank)
                     title={Boltzon Sampling},
                % the dissertation subtitle (which can    be blank)
                  subtitle={Sampling from semi-distinguishable photons},
                % the year of submission
                      year={\year} ]{dissertation}

\begin{document}

% =============================================================================

% This macro creates the standard UoB title page by using information drawn
% from the document class (meaning it is vital you select the correct degree 
% title and so on).

\maketitle

% After the title page (which is a special case in that it is not numbered)
% comes the front matter or preliminaries; this macro signals the start of
% such content, meaning the pages are numbered with Roman numerals.

\frontmatter

% This macro creates the standard UoB declaration; on the printed hard-copy,
% this must be physically signed by the author in the space indicated.

\makedecl

% LaTeX automatically generates a table of contents, plus associated lists 
% of figures, tables and algorithms.  The former is a compulsory part of the
% dissertation, but if you do not require the latter they can be suppressed
% by simply commenting out the associated macro.

\tableofcontents
\listoffigures
\listoftables
\listofalgorithms
\lstlistoflistings

% The following sections are part of the front matter, but are not generated
% automatically by LaTeX; the use of \chapter* means they are not numbered.

% -----------------------------------------------------------------------------

\chapter*{Abstract}

Since their original proposal in the 70s and 80s, quantum computers have evolved from an interesting theoretical concept to a physically realisable technology. This has been particularly exemplified with a recent publication by Arute et al.~(\emph{Nature}, 574(7779):505-510, 2019), which has for the first time demonstrated a quantum computer solving a problem that a classical computer cannot.

But while this is much cause for celebration and interest, the work towards showing what a quantum computer can really do is still yet to come. Arute et al.'s result shows a quantum computer solving a hard problem, but not a useful one, nor doing it well.

This is the question we push towards in this thesis: What problem with real-world applications can a quantum computer solve faster than a classical computer? We make contributions towards solving this problem in two ways:

First, an applications-focused approach: We show how a quantum computer can solve the Travelling Salesman Problem on bounded-degree graphs polynomially faster. This is achieved through applying quantum speedups to the best classical algorithms currently known.

Second, an architecture-focused approach: We consider how photon distinguishability and loss affect the near-term quantum architecture known as Boson Sampling. In doing so, we provide a way of mathematically modelling these imperfections as decoherence in a quantum circuit, via representation theory and the first quantisation. We then show how current classical simulation algorithms can be sped up by taking advantage of these imperfections, and use analysis to suggest what photonic regimes our simulator provides better performance for.

\chapter*{Summary of Changes}

{\bf A conditional section, of at most $1$ page} 
\vspace{1cm} 

Iff. the dissertation represents a resubmission (e.g., as the result of
a resit), this section is compulsory: the content should summarise all
non-trivial changes made to the initial submission.  Otherwise you can
omit it, since a summary of this type is clearly nonsensical.

When included, the section will ideally be used to highlight additional
work completed, and address criticism raised in any associated feedback.
Clearly it is difficult to give generic advice about how to do so, but
an example might be as follows:

\begin{quote}
\noindent
\begin{itemize}
\item Feedback from the initial submission criticised the design and 
      implementation of my genetic algorithm, stating ``there seems 
      to have been no attention to computational complexity during the
      design, and obvious methods of optimisation are missing within
      the resulting implementation''.  Chapter $3$ now includes a
      comprehensive analysis of the algorithm, in terms of both time
      and space.  While I have not altered the algorithm itself, I
      have included a cache mechanism (also detailed in Chapter $3$)
      that provides a significant improvement in average run-time.
\item I added a feature in my implementation to allow automatic rather
      than manual selection of various parameters; the experimental
      results in Chapter $4$ have been updated to reflect this.
\item Questions after the presentation highlighted a range of related
      work that I had not considered: I have make a number of updates 
      to Chapter $2$, resolving this issue.
\end{itemize}
\end{quote}
\chapter*{Supporting Technologies}

{\bf A compulsory section, of at most $1$ page}
\vspace{1cm} 

\noindent
This section should present a detailed summary, in bullet point form, 
of any third-party resources (e.g., hardware and software components) 
used during the project.  Use of such resources is always perfectly 
acceptable: the goal of this section is simply to be clear about how
and where they are used, so that a clear assessment of your work can
result.  The content can focus on the project topic itself (rather,
for example, than including ``I used \mbox{\LaTeX} to prepare my 
dissertation''); an example is as follows:

\begin{quote}
\noindent
\begin{itemize}
\item I used the Java {\tt BigInteger} class to support my implementation 
      of RSA.
\item I used a parts of the OpenCV computer vision library to capture 
      images from a camera, and for various standard operations (e.g., 
      threshold, edge detection).
\item I used an FPGA device supplied by the Department, and altered it 
      to support an open-source UART core obtained from 
      \url{http://opencores.org/}.
\item The web-interface component of my system was implemented by 
      extending the open-source WordPress software available from
      \url{http://wordpress.org/}.
\end{itemize}
\end{quote}
\chapter*{List of Abbreviations and Acronyms}

\begin{tabular}{lcl}
CSP                 &:     & Constraint Satisfaction Problem\\
HOM Dip &: & Hong-Ou-Mandel Dip\\
IQP Circuit &: & Instantaneous Quantum Polynomial Time Circuit\\
MZI &: & Mach-Zehnder Interferometer\\
SAT                 &:     & Boolean Satisfiability\\
TSP                 &:     & Travelling Salesman Problem\\
\end{tabular}

\chapter*{Acknowledgements}

There are many people without whom the experience of creating this thesis would have been made more challenging, if not impossible. In lieu of a perfect and complete list, which no doubt could be a full text of its own, I offer this imperfect and incomplete substitution.

At a fundamental level, I would like to thank the people who made this work possible in the first place: The Quantum Engineering Centre for Doctoral Training (QECDT) for offering me a studentship, the Quantum Engineering Technology Labs (QETLabs) for offering me a desk, and the Heilbronn Institute for Mathematical Research for funding my PhD. I also thank my supervisors -- Peter Turner and Noah Linden -- for keeping my work on track, and my other collaborators -- Ashley Montanaro, Ra\'{u}l Garc\'{i}a-Patr\'{o}n and Jelmer Renema -- for such great ideas and opportunities.

I have also worked alongside many great fellow researchers during my time in the above groups. Of particular note are my fellow QECDT students: Daniel Love, Geraint Gough, Henry Semenenko, Jason Mueller, Joe Smith, Lawrence Rosenfeld, Lucio Stefan, Martin Nicolle and Sam Holder. I also thank other theorists within QETLabs notably Stasja Stanisic, Sam Morley-Short, Sam Pallister, Will Dixon, Will McCutcheon, Lana Mineh and Oliver Thomas. It was a pleasure working with all of your and I wish you the best of luck in your future endeavours.

I also thank the many people I have worked with on public engagement, both in QECDT and QETLabs but also outside, such as in Pint of Science and with Bristol Doctoral College. I would in particular like to thank Nic Harrigan, Euan Allen, Alasdair Price, Holly Caskie, Jamie Thakrar, Sophie Stephens, Laura Veldenz, Ben Barber and Kate Oliver, for the opportunities they have given me to act a fool in public.

Of course, University is only one part of the rich tapestry that is a PhD life. Outside of University, I would like to thank the many sporting communities that I have become part of during my PhD, including but not limited to: parkrun, GoodGym, LGBT+ Fitness Class at Hamilton House, Project Awesome, Spin City and the University of Bristol Sport, Exercise \& Health team. These and many other groups have helped me change my life significantly.

My life has also changed in many other ways. I would like to thank the wonderful folk at the University of Bristol LGBT+ Society, the Bristol Students' Union Trans Students Network, transcaf and Off the Record Bristol, for the wonderful support they have given me. And a particular thank you to Iris Dinu for helping me find my direction in the first place.

Thank you to the many friends I have made or kept over the last 27 years, and the warmest thank you of all to my family, for the unconditional love, inspiration and support I have received my whole life. This wouldn't have been possible without you.

\vspace{1cm}

\emph{This thesis is dedicated to my grandmothers, Stella Mae Lewis on my mother's side and Anne Moylett on my father's side, both of whom were always proud about the fact that their grandchild was going to do a PhD. Wherever you two are now, I hope you are just as pleased today.}


% =============================================================================

% After the front matter comes a number of chapters; under each chapter,
% sections, subsections and even subsubsections are permissible.  The
% pages in this part are numbered with Arabic numerals.  Note that:
%
% - A reference point can be marked using \label{XXX}, and then later
%   referred to via \ref{XXX}; for example Chapter\ref{chap:context}.
% - The chapters are presented here in one file; this can become hard
%   to manage.  An alternative is to save the content in seprate files
%   the use \input{XXX} to import it, which acts like the #include
%   directive in C.

\mainmatter

% -----------------------------------------------------------------------------



% =============================================================================

% Finally, after the main matter, the back matter is specified.  This is
% typically populated with just the bibliography.  LaTeX deals with these
% in one of two ways, namely
%
% - inline, which roughly means the author specifies entries using the 
%   \bibitem macro and typesets them manually, or
% - using BiBTeX, which means entries are contained in a separate file
%   (which is essentially a databased) then inported; this is the 
%   approach used below, with the databased being dissertation.bib.
%
% Either way, the each entry has a key (or identifier) which can be used
% in the main matter to cite it, e.g., \cite{X}, \cite[Chapter 2}{Y}.

\backmatter

\bibliography{dissertation}

% -----------------------------------------------------------------------------

% The dissertation concludes with a set of (optional) appendicies; these are 
% the same as chapters in a sense, but once signaled as being appendicies via
% the associated macro, LaTeX manages them appropriatly.

\appendix

% =============================================================================

\end{document}
