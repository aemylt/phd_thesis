\chapter*{Abstract}

The Travelling Salesman Problem is one of the most famous problems in graph theory. However, little is currently known about the extent to which quantum computers could speed up algorithms for the problem. In this paper, we prove a quadratic quantum speedup when the degree of each vertex is at most $3$ by applying a quantum backtracking algorithm to a classical algorithm by Xiao and Nagamochi. We then use similar techniques to accelerate a classical algorithm for when the degree of each vertex is at most $4$, before speeding up higher-degree graphs via reductions to these instances.

Boson Sampling is the problem of sampling from the same output probability distribution as a collection of indistinguishable single photons input into a linear interferometer. 
It has been shown that, subject to certain computational complexity conjectures, in general the problem is difficult to solve classically, motivating optical experiments aimed at demonstrating quantum computational ``supremacy''.
There are a number of challenges faced by such experiments, including the generation of indistinguishable single photons.
We provide a quantum circuit that simulates bosonic sampling with arbitrarily distinguishable particles.
This makes clear how distinguishabililty leads to decoherence in the standard quantum circuit model, allowing insight to be gained.
At the heart of the circuit is the quantum Schur transform, which follows from a representation theoretic approach to the physics of distinguishable particles in first quantisation.
The techniques are quite general and have application beyond boson sampling.
%to exactly sample from this distribution on a classical computer must take superpolynomial time unless $P^{\#P} = BPP^{NP}$ and the Polynomial Hierarchy collapses to the third level. 