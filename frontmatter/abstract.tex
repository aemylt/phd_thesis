\chapter*{Abstract}

Since their original proposal in the 70s and 80s, quantum computers have evolved from an interesting theoretical concept to a physically realisable technology. This has been particularly exemplified with a recent publication by Arute et al.~(\emph{Nature}, 574(7779):505-510, 2019), which has demonstrated a quantum computer solving a problem that is believed to be classically hard.

But while this is much cause for celebration and interest, the work towards showing what a quantum computer can really do is still yet to come. Arute et al.'s result shows a quantum computer solving a hard problem, but not a useful one.

This is the question we push towards in this thesis: What problem with real-world applications can a quantum computer solve faster than a classical computer? We make contributions towards solving this problem in two ways:

First, an applications-focused approach: We show how a quantum computer can solve the Travelling Salesman Problem on bounded-degree graphs polynomially faster. This is achieved through applying a quantum speedup for Backtracking algorithms to classical algorithms for solving the Travelling Salesman Problem when the degree of the graph is at most 3 or 4. Other polynomial speedups for graphs of degree up to 6 are also shown by reduction, and speedups for arbitrary-degree graphs are believed to be possible via a recent speedup of Dynamic Programming algorithms.

Second, an architecture-focused approach: We consider how photon distinguishability and loss affect the near-term quantum architecture known as Boson Sampling. In doing so, we provide a way of mathematically modelling these imperfections as decoherence in a quantum circuit, via representation theory and the first quantisation. We then show how current classical simulation algorithms can be sped up by taking advantage of these imperfections, and use analysis to suggest what photonic regimes our simulator provides better performance for.
