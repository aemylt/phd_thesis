\chapter*{Abstract}

The Travelling Salesman Problem is one of the most famous problems in graph theory. However, little is currently known about the extent to which quantum computers could speed up algorithms for the problem. In this paper, we prove a quadratic quantum speedup when the degree of each vertex is at most $3$ by applying a quantum backtracking algorithm to a classical algorithm by Xiao and Nagamochi. We then use similar techniques to accelerate a classical algorithm for when the degree of each vertex is at most $4$, before speeding up higher-degree graphs via reductions to these instances.

Boson Sampling is the problem of sampling from the same output probability distribution as a collection of indistinguishable single photons input into a linear interferometer. 
It has been shown that, subject to certain computational complexity conjectures, in general the problem is difficult to solve classically, motivating optical experiments aimed at demonstrating quantum computational ``supremacy''.
There are a number of challenges faced by such experiments, including the generation of indistinguishable single photons.
We provide a quantum circuit that simulates bosonic sampling with arbitrarily distinguishable particles.
This makes clear how distinguishabililty leads to decoherence in the standard quantum circuit model, allowing insight to be gained.
At the heart of the circuit is the quantum Schur transform, which follows from a representation theoretic approach to the physics of distinguishable particles in first quantisation.
The techniques are quite general and have application beyond boson sampling.
%to exactly sample from this distribution on a classical computer must take superpolynomial time unless $P^{\#P} = BPP^{NP}$ and the Polynomial Hierarchy collapses to the third level. 

Boson Sampling is the problem of sampling from the same distribution as indistinguishable single photons at the output of a linear optical interferometer. 
It is an example of a non-universal quantum computation which is believed to be feasible in the near term and cannot be simulated on a classical machine.
Like all purported demonstrations of ``quantum supremacy'', this motivates optimizing classical simulation schemes for a realistic model of the problem, in this case Boson Sampling when the implementations experience lost or distinguishable photons.
Although current simulation schemes for sufficiently imperfect boson sampling are classically efficient, in principle the polynomial runtime can be infeasibly large.
In this work, we develop a scheme for classical simulation of Boson Sampling under uniform distinguishability and loss, based on the idea of sampling from distributions where at most $k$ photons are indistinguishable.
We show that asymptotically this scheme can provide a polynomial improvement in the runtime compared to classically simulating idealised Boson Sampling. 
More significantly, we show that in the regime considered experimentally relevant, our approach gives an substantial improvement in runtime over other classical simulation approaches.